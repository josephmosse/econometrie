\documentclass[A4paper]{article}

\usepackage[utf8]{inputenc}
\usepackage[T1]{fontenc}
\usepackage{textcomp}
\usepackage[french]{babel}
\usepackage{amsmath, amssymb}
\usepackage{upgreek}
\usepackage{xcolor}

\setlength\parindent{0pt}

\begin{document}
\section{Paramètres du modèle}
\subsection{Intervalle de confiance} 
Intervalle de confiance des $\upbeta_i$ : 
\begin{equation*}
\begin{split}
		1-\upalpha &= Prob\left[ t_{\upalpha/2}(n-k) \le \frac{\hat{\upbeta}_i - \upbeta_i}{\hat{\upsigma}_{\hat{\upbeta}_i}} \le  t_{1-\upalpha/2}(n-k)\right] \\
		IC^{\upbeta_i}_{1-\upalpha} &= \left[ \hat{\upbeta}_i - t_{1-\upalpha/2}(n-k) \cdot \hat{\upsigma}_{\hat{\upbeta}_i}; \hat{\upbeta}_i - t_{\upalpha/2}(n-k) \cdot \hat{\upsigma}_{\hat{\upbeta}_i}\right]
\end{split}
\end{equation*}
Intervalle de confiance des $\upbeta_i$ avec la terminologie Eviews :
\begin{equation*}
\begin{split}
		IC^{\upbeta_i}_{1-\upalpha} &= \left[ \text{Coefficient}_i- t_{1-\upalpha/2}(n-k) \cdot \text{Std. Error}_i;\text{Coefficient}_i- t_{\upalpha/2}(n-k) \cdot \text{Std. Error}_i\right] 
\end{split}
\end{equation*}
Application : \\
Ici $n-k = 248$. Le degrès de liberté du la loi de Student est $\upnu = 248 > 30$, celle ci converge donc vers la loi normale centrée réduite $T(248) \leadsto N(0,1)$. \\
D'ou : $t_{.975}(248) =  U_{.975} = 1.96$ 
\begin{align*}
		IC^{\upbeta_0}_{95\%} &= {\color{red}\left[ 40.0587 ; 70.5573 \right]} & &= \left[ 55.3080 - 1.96* 7.8026 ; 55.3080 + 1.96* 7.8026 \right]\\
		IC^{\upbeta_1}_{95\%} &= {\color{red}\left[ 0.0481 ; 0.05360 \right]} & &= \left[ 0.05084 - 1.96 * 0.0014 ; 0.05084 + 1.96 * 0.0014\right]\\
		IC^{\upbeta_2}_{95\%} &= {\color{red}\left[ 0.0023 ; 0.0024 \right]} & &= \left[ 0.0024 - 1.96 * 0 ; 0.0024 + 1.96 * 0\right]\\
		IC^{\upbeta_3}_{95\%} &= {\color{red}\left[ 0.0182 ; 0.02330\right]} & &= \left[ 0.02072 - 1.96 * 0.0013;0.02072 + 1.96 * 0.0013\right]  
\end{align*}
\newpage
\subsection{Test de signification}
Testons a présent la significativité des paramètres \\
\textbf{Spécification du test :} \\
$\text{H}_0 : \upbeta_i = 0$ Le paramètre n'est pas significatif\\
$\text{H}_1 : \upbeta_i \neq 0$ Le paramètre est significatif \\
\textbf{Statistique de test :}
\begin{equation*}
		t_{c_i} = \frac{\hat{\upbeta}_i}{\hat{\upsigma}_{\upbeta_i}}
\end{equation*}
\textbf{RDD : } \\
Si $\left| t_{c_i} \right|  < t_{.95}(n-k)$ On accepte $H_0$ au risque de $\upalpha$\\
Si $\left| t_{c_i} \right|  \geq t_{.95}(n-k)$ On rejette $H_0$ au risque de $\upalpha$		

\textbf{Application :} \\
On peut directement lire le student calculé sur Eviews : $t_c = \text{t-Statistic}$. Avec $t_{.975}(248) =  U_{.975} = 1.96$ 
\begin{align*}
		t_{c_0} &= 7.1088 & t_{c_1} &= 35.9897 & t_{c_2} &= 64.4521 & t_{c_3} &= 16.0143
\end{align*}
\textbf{Décision : } \\
$\left| t_{c_i} < t_{.975}(248) \right| \; \forall i \in [0;3]$ \; On rejette H$_0$ au risque de $\upalpha = 5\%$. Tous les paramètres estimés sont significatifs.

\newpage
\section{Analyse de la variance}
L'équation de la variance est telle que : 
\begin{equation*}
\begin{split}
		\sum_t \left( Y_t - \overline{Y}\right)^2 &= \sum_t \left( \hat{Y} - \overline{Y} \right)^2 + \sum_t e_t^2 \\
		SCT &= SCE + SCR
\end{split}
\end{equation*}
\subsection{Table d'ANOVA}
\begin{tabular}{|c|l|l|c}
\hline
\begin{tabular}[c]{@{}c@{}}Source\\ de variation\end{tabular} & \multicolumn{1}{c|}{\begin{tabular}[c]{@{}c@{}}Somme\\ des carrés\end{tabular}} & \multicolumn{1}{c|}{\begin{tabular}[c]{@{}c@{}}Degré\\ de liberté\end{tabular}} & \multicolumn{1}{c|}{\begin{tabular}[c]{@{}c@{}}Carrés\\ Moyens\end{tabular}} \\ \hline
X                                                             & $SCE = \sum_t (\hat{Y}_t - \overline{Y})^2$                                     & $k-1 = 3$                                                                       & \multicolumn{1}{c|}{$SCE/3$}                                                 \\ \hline
Résidu                                                        & $SCR = \sum_t e^2_t$                                                            & $n-k = 248$                                                                     & \multicolumn{1}{c|}{$SCR/248$}                                               \\ \hline
Total                                                         & $SCT = \sum_t(Y_t - \overline{Y})^2$                                            & $n-1 = 251$                                                                     & \multicolumn{1}{l}{}                                                         \\ \cline{1-3}
\end{tabular}
\subsection{Test de signification du coefficient de détermination}

\newpage
\section{Tests d'autocorrélation}
\subsection{Test de Durbin Watson}
On pose : 
\begin{equation*}
    \varepsilon_t = \uprho \varepsilon_{t+1} + \eta_t \qquad \eta_t \sim \text{BB}
\end{equation*}
\textbf{Spécification du test :} \\
H$_0 : \uprho = 0$ Abscence d'autocorrélation d'odre 1 \\
H$_1 : \uprho \neq 0$ Autocorrélation d'odre 1 \\
\textbf{Statistique de test :}
\begin{equation*}
    \text{DW} = \frac{\sum\left( \varepsilon_{t+1} - \varepsilon_t \right)^2 }{\sum e_t^2}
\end{equation*}
\textbf{RDD : }
\begin{table}[h!]
\begin{tabular}{ccccc}
\hline
\multicolumn{1}{|c|}{$0 ; d_1$} & \multicolumn{1}{c|}{$d_1 ; d_2$}& \multicolumn{1}{c|}{$2$} & \multicolumn{1}{c|}{$4-d_2 ; 4-d_1$} & \multicolumn{1}{c|}{$4-d_1 ; 4$}                                                                      \\ \hline
\multicolumn{1}{|c|}{\begin{tabular}[c]{@{}c@{}}autocorrelation \\ positive \\ d'ordre 1\end{tabular}} & \multicolumn{1}{c|}{\begin{tabular}[c]{@{}c@{}}Zone de doute\\ $\leftarrow$\end{tabular}} & \multicolumn{1}{c|}{Abscence d'autocorrelation (H0 vrai)} & \multicolumn{1}{c|}{\begin{tabular}[c]{@{}c@{}}Zone de doute\\ $\rightarrow$\end{tabular}} & \multicolumn{1}{c|}{\begin{tabular}[c]{@{}c@{}}autocorrelation \\ négative\\ d'ordre 1\end{tabular}} \\ \hline
\end{tabular}
\end{table} \\
\textbf{Application :}\\
\begin{equation*}
		\text{DW} = 1.2092 \qquad d_1 = 1.00 \; ; \; d_2 = 1.68
\end{equation*}
\textbf{Décision :}\\
DW$\in [d_1;d_1]$ Zone de doute

\newpage
\subsection{Test d'autocorrélation des résidus d'ordre supérieur a 1}
\textbf{Modèle :}
\begin{equation*}
		\varepsilon_t = \phi_1 \varepsilon_{t-1} + \phi_2 \varepsilon_{t-2} + \ldots + \phi_k \varepsilon_{t-k} + \eta_t \qquad \eta_t \sim \text{BB}
\end{equation*}
\textbf{Spécification du test: } \\
H$_0 : \phi_1 = \phi_2 = \ldots = \phi_k = 0$ : Non autocorrélation des résidus d'ordre $1$ à $k$. \\
H$_1$ : Au moins un $\phi_i \neq 0$ : Autocorrélation d'ordre $i$ \\
\textbf{Statistique de test :}
\begin{equation*}
		\text{Q}_{\text{stat}} = n(n+2) \sum_{k=1}^K \frac{r_k^2}{n-k} \sim \upchi^2(K)
\end{equation*}
Où $K$ est le nombre de retards \\
 \textbf{RDD}\\
Si $\text{Q}_{\text{stat}} < \upchi^2_{.95}(K)$ On accepte $H_0$ au risque de $\upalpha = 5\%$\\
Si $\text{Q}_{\text{stat}} \geq \upchi^2_{.95}(K)$ On rejette $H_0$ au risque de $\upalpha = 5\%$ \\
\textbf{Application :} \\
Grace au corrélograme (ACF) on obtient les $r_k$. Avec $K=2$ retards on a : 
\begin{equation}
		\text{Q}_{\text{stat}} = 252 * (252+2) \left[ \frac{0.386^2}{252-1} + \frac{0.247^2}{252-2} \right] = 53.521
\end{equation}
\textbf{Décision : }\\
$\text{Q}_{\text{stat}} > \upchi^2(2) = 5.991$ On rejette H$_0$ au risque de $5\%$, il y a autocorrélation des résidus a l'ordre 2.
\newpage
\section{Tests de normalité}
\subsection{Test de Jarque-Bera}
\textbf{Spécification du test :}\\
H$_0$ : La distribution des résidus est normale. \\
H$_1$ : La distribution des résidus est non normale\\
\textbf{Statistique de test :}
\begin{equation*}
	\text{JB} = \frac{n}{6} \text{SKEW} + \frac{n}{24} (\text{KURT} -3)^2 \sim \upchi^2(2)		
\end{equation*}
\textbf{RDD :} \\
Si  $\text{JB} < \upchi^2_{.95}(2)$  On accepte  H$_0$  au risque de  $\upalpha = 5\%$ \\
Si  $\text{JB} \ge  \upchi^2_{.95}(2)$  On rejette  H$_0$  au risque de  $\upalpha = 5\%$  \\
\textbf{Décision :} \\
JB $= 4.2806 < \upchi^2(2) = 5.9915$ On accepte H$_0$ au risque de $\upalpha = 5\%$, il y a distribution normale des résidus.
\subsection{Tests de de Skewness et Kurtosis}
\subsubsection{Test de Skewness (symétrie)}
\textbf{Spécification du test :} \\
H$_0$ : Symétrie normale \\
H$_1$ : Symétrie anormale \\
\textbf{Statistique de test :}
\begin{equation*}
		\frac{\upbeta_1^{1/2}}{\sqrt{6/n}}=  \frac{\text{SKEW}}{\sqrt{6/n}} \sim N(0,1)
\end{equation*}
\textbf{RDD :} \\
Si $\left| \frac{\upbeta^{1/2}_1}{\sqrt{6/n}} \right|  < 1.96$ On accepte $H_0$ au risque de $\upalpha = 5\% $\\
Si $\left| \frac{\upbeta^{1/2}_1}{\sqrt{6/n}} \right|  \geq 1.96$ On rejette $H_0$ au risque de $\upalpha = 5\%$ \\
\textbf{Application :} SKEW $= -0.2102165$
\begin{equation*}
	\frac{\text{SKEW}}{\sqrt{6/n}} = \frac{-0.2102}{\sqrt{6/252}} = -1.3622
\end{equation*}
\textbf{Décision :}\\
$\left| \frac{\upbeta^{1/2}_1}{\sqrt{6/n}} \right| = 1.3622 < 1.96 $ On accepte H$_0$ au risque de  $\upalpha=5\%$, il y a symétrie normale des résidus.
\newpage
\subsubsection{Test de Kurtosis (applatissement)}
\textbf{Spécification du test :} \\
H$_0$ : Applatissemnt normal \\
H$_1$ : Exès de kurtosis \\
\textbf{Statistique de test :}
\begin{equation*}
		\frac{\upbeta_2-3}{\sqrt{24/n}}=  \frac{\text{KURT}}{\sqrt{24/n}} \sim N(0,1)
\end{equation*}
\textbf{RDD :} \\
Si $\left| \frac{\upbeta_2}{\sqrt{24/n}} \right|  < 1.96$ On accepte $H_0$ au risque de $\upalpha = 5\% $\\
Si $\left| \frac{\upbeta_2}{\sqrt{24/n}} \right|  \geq 1.96$ On rejette $H_0$ au risque de $\upalpha = 5\%$ \\
\textbf{Application :} SKEW $= 2.5195 $
\begin{equation*}
	\frac{\text{KURT}}{\sqrt{24/n}} = \frac{2.5195}{\sqrt{24/252}} = -1.557
\end{equation*}
\textbf{Décision :}\\
$\left| \frac{\upbeta_2}{\sqrt{24/n}} \right| = 1.557 < 1.96 $ On accepte H$_0$ au risque de  $\upalpha=5\%$, il y a applatissement normal des résidus.

\newpage
\section{Tests d'homoscédasticité}
\subsection{Test de ARCH}
\begin{equation*}
		\varepsilon^2_t = \upbeta_0 + \upbeta_1 \varepsilon^2_{t-1} + \upbeta_2 \varepsilon^2_{t-2} + \ldots + \upbeta_k \varepsilon^2_{t-k} + \eta_t \qquad \eta_t \sim N(0,\upsigma_n)
\end{equation*}
\textbf{Spécification du test :} \\
H$_0$ : $beta_1 = \upbeta_2 = \ldots = \upbeta_k = 0$ homoscédasticité \\
H$_1$ : au moins un $\upbeta_i \neq 0$ hétéroscédasticité \\
\textbf{Statistique de test :}
\begin{equation*}
		nR^2 \sim \upchi^2_{1-\upalpha}(k)
\end{equation*}
Où $k$ est le nombre de retards \\
\textbf{RDD :} \\
Si $nR^2 < \upchi^2_{1-\upalpha}(k)$ On accepte $H_0$ au risque de $\upalpha = 5\%$\\
Si $nR^2 \geq \upchi^2_{1-\upalpha}(k)$ On rejette $H_0$ au risque de $\upalpha = 5\%$ \\
\textbf{Application :}
\begin{equation*}
	nR^2 = 252 * \quad = \quad \qquad \upchi^2_{.95}(\quad) = \quad 
\end{equation*}
\textbf{Décision : }
$nR^2 > \upchi^2_{.95}$ On accepte H$_0$ au risque de $\upalpha = 5\%$,
\newpage
\subsection{Test de Goldfeld-Quandt}
\textbf{Spécification du test :} \\
H$_0$ : Homoscédasticité \\
H$_1$ : Hétéroscédasticité \\
\textbf{Statistique de test :}
\begin{equation*}
	F_c = \frac{\text{SCR}_2}{\text{SCR}_1} \sim F_{1-\upalpha} \left( \frac{n-m-2k}{2} ; \frac{n-m-2k}{2} \right) 
\end{equation*}
\textbf{RDD :} \\
Si $ F_c < F_{1-\upalpha}\left( \frac{n-m-2k}{2} ; \frac{n-m-2k}{2} \right) $ On accepte $H_0$ au risque de $\upalpha$\\
Si $ F_c \geq F_{1-\upalpha}\left( \frac{n-m-2k}{2} ; \frac{n-m-2k}{2} \right) $ On rejette $H_0$ au risque de $\upalpha$ \\
Où m est le nombre d'observations omises et k le nombre de paramètres estimés. \\
\textbf{Application : }\\
Si pour $n=30$, on supprime $m=8$ observations, alors pour $n=252$ observations, on supprime : $m = \frac{258*8}{30} \approx 66$ observations.\\
En regressant les deux sous échantillons, on obtient : \\
$SCR_1 = 6848.016 \text{ et } SCR_2 = 8914.091$  0.7682237 
\begin{equation*}
\begin{split}
	&F_c = \frac{\text{SCR}_2}{\text{SCR}_2} = \frac{8914.091}{6848.016} = 0.7682237 \\ 
	&F_{.95}(\frac{252-66-8}{2},\frac{252-66-8}{2}) = F_{.95}(89,89) = 1.42 
\end{split}
\end{equation*}
\textbf{Décision : }\\
$F_c < F_{lu} $ On accepte H$_0$ au risque de $\upalpha = 5\%$, il y a homoscédasticité.
\newpage
\subsection{Test de White}
\begin{equation*}
		e_t^2 = \upalpha_0 + \upalpha_1 X_t + \upalpha_2 X_t^2 + \eta_t
\end{equation*}
\textbf{Spécification du test :} \\
H$_0$ : $\upalpha_0 = \upalpha_1 =\upalpha_2 = 0$ Homoscédasticité \\
H$_1$ : Au moins un $\upalpha_i \neq 0 \quad \forall i \in (0,1,2)$ Hétéroscédasticité \\
\textbf{Statistique de test :}
\begin{equation*}
		\text{W} = nR^2 \sim \upchi^2(K)
\end{equation*}
Où $K$ est me nombre de paramètres estimés.\\
\textbf{RDD : }\\
Si $nR^2 < \upchi^2_{1-\upalpha}(K)$ On accepte $H_0$ au risque de $\upalpha$.\\
Si $nR^2 \geq \upchi^2_{1-\upalpha}(K)$ On rejette $H_0$ au risque de $\upalpha$.\\
\textbf{Application : }\\
En termes non croisés :
\begin{equation*}
	\text{W} = 0.1744 * 252 = 43.9468
\end{equation*}
En termes croisés :
\begin{equation*}
	\text{W} = 0.1835 * 252 = 46.2424
\end{equation*}
\textbf{Décision :}\\
\begin{itemize}	
	\item On a estimé en termes non croisés $6$ paramètres, donc \\
	$\upchi^2_{.95}(6)=12.592 < \text{W}$, on rejette H$_0$ au risque de $\upalpha = 5\%$, il y a hétéroscédasticité.
	\item On a estimé en termes croisés $9$ paramètres, donc \\
	$\upchi^2_{.95}(9)=16.919 < \text{W}_c$, on rejette H$_0$ au risque de $\upalpha = 5\%$, il y a hétéroscédasticité.
\end{itemize}
\newpage
\subsection{Test de Glejser}
\begin{equation*}
	|e_t| = a_0 + a_1 +X_{kt}
\end{equation*}
Il y a autant de régressions que de variables explicatives\\ 
\textbf{Spécification du test :}\\
H$_0 : a_1 = 0$\\
H$_1 : a_1 \neq 0$\\
\textbf{Statistique de test :}
\begin{equation*}
		t_c = \frac{\hat{a_1}}{\hat{\upsigma}_{\hat{a}_1}} \sim T(n-2)
\end{equation*}
\textbf{RDD : } \\
Si $|t_c|< t_{1-\upalpha/2}$ On accepte $H_0$ au risque de $\upalpha$ (homoscédasticité)\\
Si $|t_c| \geq t_{1-\upalpha/2}$ On rejette $H_0$ au risque de $\upalpha$ (hétéroscédasticité)\\
\section{Robustesse du modèle}
\subsection{Validité du modèle}
\subsubsection{Test de signification des paramètres}
\subsection{Test de signification des coefficients de détermination}
\subsubsection{Test de Durbin Watson}
\subsection{Stabilité du modèle}
\subsubsection{Test d'ANACOVA}
\subsubsection{Test de comparaison des deux coefficients de regression}
\subsubsection{Test de comparaison de 2 coefficients de correlation}



\end{document}
